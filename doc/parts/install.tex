\subsection{Linux}

The Linux versions use CMake for their build system.

To compile the thumbnailer, a build folder needs to be created.
Open a terminal in the build folder and execute the CMake command.

\begin{lstlisting}
cmake -DCMAKE_BUILD_TYPE=Release <path to linux folder in project>
\end{lstlisting}

With this command it will configure for the GNOME and KDE versions.

To only build the GNOME or KDE version, use these options to turn one of the platforms off:
\begin{lstlisting}
-Dkde=OFF or -Dgnome=OFF
\end{lstlisting}

After it is done configuring, the \emph{make} command can be executed to compile it.
When it is compiled, the \emph{make test} command can be used to run the unit-tests or
the \emph{make install} can to install it system-wide.

After the install command, it will work in GNOME immediately, in KDE the Dolphin plugin needs to be enabled first.

To enable it open Dolphin and go to \emph{Control > Configure Dolphin > General > Previews}
and enable previews for HDF5 Files.

If the thumbnailer should be uninstalled again, \emph{make uninstall} needs to be executed in the build folder.

\subsection{Windows}

The Windows implementation is a VisualStudio solution.

It consists of 2 sub projects. One is the thumbnailer itself,
the other one is a \href{https://marketplace.visualstudio.com/items?itemName=VisualStudioProductTeam.MicrosoftVisualStudio2017InstallerProjects}{installer project}
to create an install wizard.

To install the thumbnailer, either the installer can be used (Recommended) or it can be installed manually,
by executing the \emph{regsvr32.exe} command as an admin on the dll.

\subsection{OSX}

The OSX implementation is a Xcode project.

When you build it, it automatically installs itself for the current user, into the directory
\emph{/Users/<username>/Library/QuickLook/HDF5 Thumbnailer.qlgenerator}.

To make it available to all users of the system, copy the \emph{HDF5 Thumbnailer.qlgenerator} folder into
\emph{/Library/QuickLook/}

After copying it into one of the folders, run \emph{qlmanage -r} or restart the system to reload QuickLook.