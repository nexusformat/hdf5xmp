\subsection{Python scripts}
The two Python scripts are called \emph{thumbnailInserter.py} and \emph{metadataReader.py}
They require the \emph{argparse} and \emph{xmltodict} modules to be installed.

\subsubsection{thumbnailInserter.py}
This script takes a HDF5 file, and writes a version of it with the thumbnail and metadata.
It has also the possibility of updating or delete existing data.

When working with an HDF5 file that doesn't contain any XMP block, a new one will be created.
If the file already has XMP data inside the script will update the data you give it and leave the rest alone.

To work the script needs a few arguments. The first argument is the HDF5 file you want to use.
If the HDF5 file doesn't already have a thumbnail, we need to provide the path to an image file
via the \textbf{\emph{--img}} argument. The format of the image can be any format understandable by the system,
but generally PNG or JPG for biggest possible compatibility and good quality/fileSize.

If you want to write the data into a sidecar file, you need to provide the \textbf{\emph{--sidecar}} argument.

When you don't provide the sidecar argument you need to specify an \emph{outFile}
where to write the file with the metadata and thumbnail.
This is specified with the \textbf{\emph{--outFile}} argument. (Or \textbf{\emph{-o}} as a short version).
If omitted when writing to a sidecar file the name of the outFile will be the name of the
HDF5 file with the \emph{hdf5} suffix replaced with \emph{xmp}.


To provide additional data to insert into you can use the \textbf{\emph{--data}} or \textbf{\emph{-d}} argument.
You can provide any amount of those arguments. The first word written after the argument
represents the key name of the data and the second one the data to store in it.

There are few restrictions as to what can be used as key names and values.
Key names can't contain any character not normally allowed in an XML tag and can't be called \emph{Thumbnails}.
Values can contain any printable character except \&, < and >.
If you want to use any of these they need to be escaped. \emph{None} is also not allowed as a saved value.

To remove a key-Value pair when updating, just set the value to \emph{None}

Here is an example of inserting an image and some data into a file.
\begin{lstlisting}
./thumbnailInserter.py oldFiles/engine.hdf5 \
    -o newFiles/engine.hdf5 --img ~/Pictures/placeholder.png \
    -d Author "Bill from nextdoor" -d ApprovedBy Me
\end{lstlisting}
In this example we insert \emph{placeholder.png} as the thumbnail,
\emph{Bill from nextdoor} as a value called \emph{Author} and who the file got approved by.

If for example the ApprovedBy field isn't needed anymore and we want to remove it
we can set its value to \emph{None} to remove the field.

\begin{lstlisting}
./thumbnailInserter.py newFiles/engine.hdf5 \
    -o newerFiles/engine.hdf5 -d ApprovedBy None
\end{lstlisting}
This removes the \emph{ApprovedBy} field without touching the inserted thumbnail or Author.

\subsubsection{metadataReader.py}
This script reads the XMP data from the specified HDF5 or XMP file.

The script has the option to output the thumbnail to a file or print the metadata to stdout formatted as JSON.

\begin{lstlisting}
./metadataReader.py newerFiles/engine.hdf5
\end{lstlisting}
This will output: \textbf{\{"Author": "Bill from nextdoor"\}}