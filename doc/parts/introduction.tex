\subsection{Purpose}

The purpose of this document is to describe the functionality of the HDF5 thumbnailer and how to use it.
It should also tell how to add your own data to the HDF5 userblock without conflicting the current XMP metadata.

\subsection{Definitions}

\subsubsection{XMP}
XMP is a format by Adobe, created to store metadata and thumbnails, and to be embedded into different file types.
The XMP data can either be stored in a sidecar file next to the HDF5 file or directly in the file's userblock.

The Python scripts allow you to specify additional data to populate the XMP with when inserting or updating the data.
The scripts can also read that data from the XMP file again in JSON format for further use.

\subsection{System overview}
The purpose of this thumbnailer is to generate thumbnails for HDF5 files by extracting them from XMP data.
The XMP data is either stored in a sidecar file or embedded into the HDF5 userblock, prefixed by a signature.

The use of that signature should also be there to make storage of any data in the userblock more unified safer from overriding.
A more detailed description about the signature can be found later in the document.

\newpage
\subsection{Source structure}

\dirtree{%
.1 /.
.2 common \DTcomment{Core functionality for extracting a thumbnail}.
.2 doc \DTcomment{Documentation}.
.2 examples \DTcomment{Example HDF5 and XMP files with varying thumbnail store methods}.
.2 hdf\_python \DTcomment{Python scripts for inserting, reading and updating metadata/thumbnails}.
.3 tests \DTcomment{Unit tests for the Python scripts}.
.2 linux \DTcomment{Linux implementations for the thumbnailer}.
.3 gnome \DTcomment{GNOME implementation for Nautilus}.
.3 kde \DTcomment{KDE implementation for Dolphin}.
.3 tests \DTcomment{Unit-tests for the core thumbnailer functionality}.
.2 windows \DTcomment{Windows implementation for the thumbnailer (Visual Studio project)}.
.2 osx \DTcomment{OSX implementation for the thumbnailer (Xcode project)}.
}
In the section Components~\ref{sec:Components} the important parts will be explained further.
