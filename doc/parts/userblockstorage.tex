\subsection{Overview}
In this section will be explained how the XMP data is stored in the HDF5 file's userblock
and how to store your own data without damaging other data stored there with this method.

\subsubsection{How the userblock works}
The userblock is a arbitrarily sized section that can be placed in any HDF5 file before it starts.

When reading a HDF5 file the HDF5 library checks for the HDF5 signature at the start of the file.
If it doesn't find a signature then it assumes that it contains a userblock and then it checks at the
\emph{next power of 2} for the signature.
It will do this until it either finds a signature or reaches the end of the file.
When writing to the userblock it needs to pad the file until it reaches the next power of 2 for the HDF5 signature.


\subsection{Position in the userblock}
To ensure compatibility between multiple data blocks a signature
to be used for the stored data blocks is specified.
If the userblock has a correct signature then additional data can be inserted right before
or after any other data block in the userblock.

\subsection{Signature}
The signature for the data blocks is an extension of the HDF5 signature with an added content length.

The HDF5 signature is made up of 8 bytes with the following content:

\begin{tabular}{ |r|c|c|c|c|c|c|c|c| }
    \hline
    Decimal:     & 137         & 72 & 68 & 70 & 13        & 10        & 26          & 10 \\ 
    \hline
    Hexadecimal: & 89          & 48 & 44 & 46 & 0d        & 0a        & 1a          & 0a \\ 
    \hline
    ASCII:       & \verb+\211+ & H  & D  & F  & \verb+\r+ & \verb+\n+ & \verb+\032+ & \verb+\n+ \\ 
    \hline
\end{tabular}

The signature for a custom data block is made up of two parts.

The first is the identifier:

\begin{tabular}{ |r|c|c|c|c|c|c|c|c| }
    \hline
    Decimal:     & 137         & 72 & ** & ** & 13        & 10        & 26          & 10 \\ 
    \hline
    Hexadecimal: & 89          & 48 & ** & ** & 0d        & 0a        & 1a          & 0a \\ 
    \hline
    ASCII:       & \verb+\211+ & H  & *  & *  & \verb+\r+ & \verb+\n+ & \verb+\032+ & \verb+\n+ \\ 
    \hline
\end{tabular}

Except for bytes 3 and 4 it is the same as the HDF5 identifier.
Bytes 3 and 4 can be any value as long as it isn't used by another program, meaning they can't be the same
as the ones in the identifiers for the HDF5 and XMP metadata signature.

For example, the XMP block with the metadata and thumbnail have this signature:

\begin{tabular}{ |r|c|c|c|c|c|c|c|c| }
    \hline
    Decimal:     & 137         & 72 & 77 & 80 & 13        & 10        & 26          & 10 \\
    \hline
    Hexadecimal: & 89          & 48 & 4D & 50 & 0d        & 0a        & 1a          & 0a \\
    \hline
    ASCII:       & \verb+\211+ & H  & M  & P  & \verb+\r+ & \verb+\n+ & \verb+\032+ & \verb+\n+ \\
    \hline
\end{tabular}

The \textbf{M}etadata XM\textbf{P} header is extracted by the thumbnailer, while others are skipped.

The next part of the signature is 8 bytes describing the size of the data, in \textbf{big-endian} byte order.
Storing the size lets the thumbnail inserter know where it could potentially insert the XMP block and
it lets the thumbnailer find the data faster 

Here is an example size of 105585 bytes:

\begin{tabular}{ |r|c|c|c|c|c|c|c|c| }
    \hline
    Hexadecimal: & 00 & 00 & 00 & 00 & 00 & 01 & 9C & 71 \\
    \hline
\end{tabular}

With the identifier, our XMP data will have this signature:

\begin{tabular}{ |r|c|c|c|c|c|c|c|c|c|c|c|c|c|c|c|c| }
    \hline
    Hexadecimal: & 89 & 48 & 4D & 50 & 0D & 0A & 1A & 0A & 00 & 00 & 00 & 00 & 00 & 01 & 9C & 71 \\
    \hline
\end{tabular}

When the thumbnailer reads the signatures it first reads the first 8 bytes.
It then looks if it's a valid header by just ignoring byte 3 and 4.
If it's valid then it checks if it's an XMP with metadata by comparing it to the HMP header.